\documentclass{beamer}

\usetheme{Szeged}
\usecolortheme{dolphin}

\usepackage[T1]{fontenc}
\usepackage[utf8]{inputenc}
\usepackage{lmodern}
\usepackage[ngerman]{babel}
\usepackage{graphicx}
\usepackage{fontawesome5} % social media icons
\usepackage{ccicons}

\usepackage{csquotes} % needed by biblatex for german quotes
\usepackage[maxbibnames=99, % print all author names in bibliography
            citestyle=numeric]{biblatex}
\addbibresource{references.bib}
\setbeamertemplate{bibliography item}{\insertbiblabel}
\renewcommand*{\bibfont}{\scriptsize}

\usepackage{tikz}
\usetikzlibrary{calc,angles,decorations.pathmorphing,shapes}

\beamertemplatenavigationsymbolsempty

\title[Sicherer Radfahren \hspace*{4cm} \insertframenumber/\inserttotalframenumber]{Sicherer Radfahren}
\subtitle{Wie offene Unfalldaten helfen können}
\author[Dr.\,Thomas Terstiege]{Dr.\,Thomas Terstiege}
\institute[Dr.\ Thomas Terstiege, Code for Münster]{Code for Münster}
\date{26. September 2019\\{\scriptsize{}Forum Citizen Science 2019}}%\\Universität Münster, Institut für Geoinformatik\par}}

\begin{document}

\frame{
  \titlepage
  \centering
  	\ccLogo\ccAttribution\ccShareAlike\\
		{\tiny Dieses Werk ist lizenziert unter einer "`Creative Commons Namensnennung - Weitergabe unter gleichen Bedingungen 4.0 International Lizenz"' (CC BY-SA 4.0).\par}
}

\section{Einleitung}
\subsection{Einleitung}

\begin{frame}
  \centering
  
  \includegraphics[width=\textwidth]{img/WN-screenshot.png}
  
  \vfill
  {\scriptsize (Bildquelle: \cite{Kalitschke2019})}
\end{frame}

\begin{frame}
  \frametitle{Entwicklung Unfälle mit Personenschaden}

  \centering
  
  \resizebox{\linewidth}{!}{
    \only<1>{\input{img/anzahl-unfaelle-pro-jahr1}}
    \only<2>{\input{img/anzahl-unfaelle-pro-jahr2}}
  }
  
    {\scriptsize (Grafik inspiriert durch: \cite[Folie~4]{Brockmann2017})\par}
\end{frame}

\subsection{genauerer Blick in die Daten}

% \begin{frame}
%   \frametitle{Entwicklung Unfallbeteiligte}
%   \centering
%   \includegraphics[width=0.925\textwidth]{img/anteil-unfaelle-beteiligte.png}
%   \begin{tikzpicture}[overlay,remember picture]
% 	  \visible<2->{
% 	    \node at ($(current page.center) + (1.0, 0.3)$) {\color{red}\footnotesize Rad-PKW, 2016--2018: 32\%};
% 	  }
% 	  \visible<2->{
%   	  \node at ($(current page.center) + (1.0, -0.7)$) {\color{red}\footnotesize PKW-PKW, 2016--2018: 20\%};
%   	}
%   	\visible<3->{
%   		\node at ($(current page.center) + (1.0, -1.1)$) {\color{red}\footnotesize Rad alleine, 2016--2018: 9\%};
%   	}
%   	\visible<3->{
%       \node at ($(current page.center) + (1.0, -1.6)$) {\color{red}\footnotesize Rad-Rad, 2016--2018: 9\%};
%     }
% 	\end{tikzpicture}
%   \vfill
%   {\scriptsize (Bildquelle: \cite[S.~11]{Baier2018})\par}
% \end{frame}

\begin{frame}
  \frametitle{Entwicklung UnfallverursacherInnen (Quelle: \cite{Baier2018})}
  \centering  
  \includegraphics[width=0.925\textwidth]{img/hauptverursacher.png}
  \begin{tikzpicture}[overlay,remember picture]
	  \visible<2->{
      \node at ($(current page.center) + (1.5, 1.2)$) {\color{red}\footnotesize PKW, 2016--2018: 58\%};
    }
    \visible<5->{
      \node at ($(current page.center) + (1.45, 0.9)$) {\color{red}\footnotesize davon Alleinunfall: < 3\%};
    }
	  \visible<3->{
  	  \node at ($(current page.center) + (0.5, -0.9)$) {\color{red}\footnotesize Rad, kein Alleinunfall, 2016--2018: 20\%};
  	}
	  \visible<4->{
      \node at ($(current page.center) + (1.0, -1.95)$) {\color{red}\footnotesize Rad, Alleinunfall, 2016--2018: 9\%};
    }
	\end{tikzpicture}
  \vfill
  {\scriptsize (Bildquelle: \cite[S.~13]{Baier2018})\par}
\end{frame}

\section{Vorstellung "`Crashes"'}

\subsection{Zugänglichkeit Unfalldaten}

\begin{frame}
  \frametitle{\subsecname}
  
  \begin{itemize}
    \item Verkehrsunfallstatistik der Polizei \cite{Polizei2019}
    \pause
    \item Studie der Unfallforschung der Versicherer \cite{Baier2018}
    \pause
    \item nicht interaktiv
  \end{itemize}
\end{frame}

\begin{frame}
  \frametitle{Datengrundlage für \href{https://crashes.codeformuenster.org}{\texttt{crashes.codeformuenster.org}}}
  
  \begin{itemize}
    \item Rohdaten aller Unfälle von der Polizei Münster 2007-2018
    \begin{itemize}
      \item 1 Excel-Tabelle pro Jahr
    \end{itemize}
    \pause
    \item Gerald hat Daten aufwändig bereinigt, geokodiert (Großteil) und als Datenbank zugänglich gemacht
  \end{itemize}  
\end{frame}

\subsection{https://crashes.codeformuenster.org}

\begin{frame}[plain]
  \frametitle{\url{https://crashes.codeformuenster.org}}
  
  \includegraphics[width=1.1\textwidth]{img/einfuehrung-crashes.png}
  
  \begin{tikzpicture}[overlay,remember picture]
  
    \visible<2->{
      \node[align=left] at ($(current page.center) + (-3.25, 1.2)$) {\fcolorbox{red}{white}{\parbox{0.32\linewidth}{interaktive Karte mit leicht erkennbaren Unfallhäufungsstellen}}};
    }
    
    \visible<3->{
      \node at ($(current page.center) + (-2.0, -3.5)$) {\fcolorbox{red}{white}{\parbox{0.32\linewidth}{umfangreiche Filtermöglichkeiten}}};
    }
    
    \visible<4->{
      \node at ($(current page.center) + (-4.25, -1.0)$) {\fcolorbox{red}{white}{Verkehrsmittel}};
    }

    \visible<5->{
      \node at ($(current page.center) + (-4.25, -2.5)$) {\fcolorbox{red}{white}{Verletzte}};
    }

    \visible<6->{
      \node at ($(current page.center) + (-1.25, -1.75)$) {\fcolorbox{red}{white}{Unfalltypen}};
    }
    
    \visible<7->{
      \node at ($(current page.center) + (1.75, -2.0)$) {\fcolorbox{red}{white}{\parbox{0.25\linewidth}{Zeitfilter (Jahr, Monat, Wochentag, Uhrzeit)}}};
    }
        
    \visible<8->{
      \node at ($(current page.center) + (4.75, -2.0)$) {\fcolorbox{red}{white}{\parbox{0.225\linewidth}{Kartenoptionen und Downloadmöglichkeit}}};
    }
  \end{tikzpicture}
\end{frame}

\begin{frame}[plain]
  \frametitle{\url{https://crashes.codeformuenster.org}}
  \centering
  
  \includegraphics[width=0.75\textwidth]{img/detailansicht.png}
  
  \begin{tikzpicture}[overlay,remember picture]
    \node[align=left] at ($(current page.center) + (-4.5, 1.0	)$) {\fcolorbox{red}{white}{\parbox{0.2\linewidth}{Detailansicht (Uhrzeit, Beteiligte, Alter, etc.)}}};
  \end{tikzpicture}
\end{frame}


\begin{frame}[plain]
  \frametitle{\url{https://crashes.codeformuenster.org}}

  \includegraphics[width=1.1\textwidth]{img/parkunfaelle-schillerstrasse.png}
  
  \begin{tikzpicture}[overlay,remember picture]
    \node[align=left] at ($(current page.center) + (-3.25, 1.2)$) {\fcolorbox{red}{white}{\parbox{0.275\linewidth}{Unfallhäufung mit parkenden Autos (2007--2018)}}};
  \end{tikzpicture}
\end{frame}

\begin{frame}
  \frametitle{Fahrradstraße?}
  
  \centering
  
  \includegraphics[width=\textwidth]{img/schillerstrasse.jpg}
  
  \vfill  
  {\scriptsize (Bildquelle: \cite{Chrobak2018})}
\end{frame}

\begin{frame}
  \frametitle{Unfalltypen sind detailliert filterbar}
  \centering
  
  \includegraphics[width=0.9\textwidth]{img/unfalltyp-details.png}
  
  \vfill
  {\scriptsize (Bildquelle: \cite[\href{https://recht.nrw.de/lmi/owa/br_show_anlage?p_id=15549}{Anlage 9}]{IMNRW2019})\par}
\end{frame}

\begin{frame}
  \frametitle{Beispielanalyse Radunfälle}
  \centering
  \begin{columns}
    \begin{column}{0.5\textwidth}
      \begin{itemize}
    \item Unfälle 2007--2018 mit Fahrradbeteiligung und Personenschaden 
    \begin{itemize}
      \item ohne PKW
      \item ohne Abbiege- oder Einbiegen/Kreuzen-Unfälle
    \end{itemize}
    \item<2> Radwege zu eng?!
  \end{itemize}
    \end{column}
    \begin{column}{0.5\textwidth}
      \includegraphics[width=\textwidth]{img/radwege-unfaelle.png}
    \end{column}
  \end{columns} 
\end{frame}

\section{Ausblick}

\begin{frame}
  \frametitle{Weitere Schritte}  
  \centering  
  
  \begin{itemize}
    \item "`Ungenauigkeiten bei der Lokalisierung der Unfälle"' \cite[S.~38]{Baier2018}
    \includegraphics[width=0.7\textwidth]{img/editor-screenshot.png}
    \item weitere Filter
    \item da freier Quelltext:
    \begin{itemize}
      \item \href{https://github.com/codeformuenster/crashes-shiny/issues}{\usebeamercolor[fg]{frametitle}Eure Ideen!}
      \item mitentwickeln: \href{https://codeformuenster.org}{\usebeamercolor[fg]{frametitle}jeden Dienstag, 18:30 Uhr, drei:klang}
    \end{itemize}
  \end{itemize}
\end{frame}

\begin{frame}
  \frametitle{Danke für die Aufmerksamkeit!}
  
  \centering
  \begin{columns}
  \begin{column}{0.67\linewidth}
      Kontakt (ich bin auf Jobsuche \faSmileWink[regular]):
      \begin{itemize}
        \item[\faEnvelope] \href{mailto:t.terstiege@posteo.de}{\texttt{t.terstiege@posteo.de}}
        \item[\faMastodon] \href{https://social.wiuwiu.de/@kartoffelsalat}{\texttt{@kartoffelsalat@social.wiuwiu.de}}
        \item[\faTwitter]  \href{https://twitter.com/krtfflslt}{\texttt{@krtfflslt}}
      \end{itemize}
    \end{column}
    \hfill
    \begin{column}{0.36\linewidth}
      Besonderen Dank an:
      \begin{itemize}
        \item Gerald Pape\\(Code for Münster)
        \item Christian Römer (Code for Münster)
        \item Tobias Bradtke (Code for Münster)
        \item ADFC Münster
        \item IG Fahrradstadt Münster
      \end{itemize}
    \end{column}
  \end{columns}

\end{frame}

\begin{frame}
  \frametitle{Quellen}
  \printbibliography[title=Quellen]
\end{frame}

\end{document}
